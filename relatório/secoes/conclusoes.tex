\section{Considerações Finais}

A análise dos dados do ENADE para Santa Catarina revelou padrões consistentes entre modalidade de ensino, categoria administrativa e os indicadores de desempenho acadêmico. Observou-se que a grande maioria dos cursos é presencial, representando cerca de 80\% do total, enquanto cursos na modalidade Educação a Distância correspondem a apenas 20\%. Isso indica que, embora o EaD esteja presente, não substituiu o ensino presencial no estado. 

No que tange à categoria administrativa, as instituições privadas com e sem fins lucrativos predominam, correspondendo a mais de 80\% do total, enquanto públicas, comunitárias e especiais representam menos de 20\%. A distribuição do IGC evidencia que a maioria das instituições privadas concentra-se em faixas intermediárias (2.2 a 3.05), enquanto algumas públicas federais alcançam as faixas mais altas (3.9 a 4.32), indicando que, apesar de sua menor quantidade, instituições públicas apresentam melhor desempenho médio. 

As variáveis contínuas (Conceito ENADE, IDD, CPC e IGC) mostram comportamento relativamente regular, com tendência central próxima à média, embora o Conceito Enade apresente leve inclinação para valores inferiores. O CPC demonstra desempenho consistente em torno da média, e o IGC indica que a maior parte das instituições possui qualidade média em relação ao país. Boxplots e tabelas de contingência reforçam a influência tanto da modalidade de ensino quanto da categoria administrativa nos resultados, evidenciando desigualdades e oportunidades de melhoria.  

Esses achados sugerem que políticas e estratégias voltadas à melhoria da qualidade acadêmica devem considerar tanto a distribuição da modalidade de ensino quanto as diferenças entre categorias administrativas, especialmente para fortalecer o desempenho das instituições privadas e ampliar a equidade entre públicas e privadas.
