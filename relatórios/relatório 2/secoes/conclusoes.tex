\section{Considerações Finais}

Este trabalho avançou na análise exploratória dos dados do ENADE de Santa Catarina, aplicando métodos de inferência estatística para testar hipóteses formuladas sobre os indicadores de qualidade. O objetivo central foi validar suposições sobre a média do Conceito Preliminar de Curso (CPC) entre modalidades, a proporção de instituições federais e a relação de dependência entre o IDD e o Conceito Enade.

A aplicação dos testes revelou um cenário de conclusões distintas. Para a \textbf{Hipótese 1}, encontrou-se uma diferença estatisticamente robusta (\textit{p-valor} $\approx 3{,}09 \times 10^{-6}$), confirmando que os cursos presenciais possuem, em média, um CPC superior aos da modalidade EaD em Santa Catarina. Em contrapartida, a \textbf{Hipótese 2} não pôde ser corroborada; o teste não forneceu evidências para rejeitar a hipótese nula, indicando que a proporção de instituições federais não é significativamente maior que os $25\%$ hipotetizados.

As análises de correlação e regressão (\textbf{Hipóteses 3 e 4}) foram particularmente elucidativas. Confirmou-se uma forte correlação positiva ($r \approx 0{,}694$) e uma influência estatisticamente significativa do IDD sobre o Conceito Enade. No entanto, é crucial ponderar a magnitude desse efeito. O coeficiente de determinação ($R^2 \approx 0{,}481$) demonstra que, embora o IDD seja um preditor relevante, ele explica menos da metade da variabilidade no Conceito Enade. Isso sugere que a significância estatística não se traduz, neste caso, em uma influência prática dominante.

A constatação de que $51{,}9\%$ da variação do Conceito Enade não é explicada pelo IDD aponta para a principal limitação deste modelo de regressão. Fica evidente a necessidade de investigações futuras que incluam outras variáveis (como as que compõem o próprio CPC: infraestrutura, qualificação docente ou a percepção discente) para construir um modelo preditivo mais completo.

Em suma, este estudo reforça a natureza probabilística da inferência estatística. A rejeição de uma hipótese nula não representa uma prova irrefutável, mas sim uma forte evidência contra $H_0$, baseada no nível de significância adotado de $5\%$. Os resultados aqui obtidos, portanto, devem ser interpretados como indicadores de tendências fortes, e não como regras determinísticas, sublinhando a importância de não equiparar significância estatística com causalidade ou relevância prática absoluta.
