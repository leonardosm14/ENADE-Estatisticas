\section{Considerações Finais}

Este trabalho avançou na análise exploratória dos dados do ENADE de Santa Catarina, aplicando métodos de inferência estatística para testar hipóteses formuladas sobre os indicadores de qualidade. O objetivo central foi validar suposições sobre a média do Conceito Preliminar de Curso (CPC) entre modalidades, a proporção de instituições federais e a relação de dependência entre o IDD e o Conceito Enade.

A aplicação dos testes revelou um cenário de resultados consistentes e estatisticamente significativos. Para a \textbf{Hipótese 1}, encontrou-se uma diferença estatisticamente robusta (\textit{p-valor} $\approx 3{,}09 \times 10^{-6}$), confirmando que os cursos presenciais possuem, em média, um CPC superior aos da modalidade EaD em Santa Catarina. Já a \textbf{Hipótese 2} também levou à rejeição da hipótese nula, uma vez que o \textit{p-valor} obtido ($2{,}66 \times 10^{-8}$) é muito inferior ao nível de significância de 5\%. Conclui-se, portanto, que a proporção de instituições federais em Santa Catarina é significativamente diferente de 25\%, sendo, na verdade, consideravelmente menor (aproximadamente 5,3\%). Esse resultado evidencia uma sub-representação das instituições federais no sistema de ensino superior do estado.

As análises de correlação e regressão (\textbf{Hipóteses 3 e 4}) reforçaram essas conclusões. Confirmou-se uma forte correlação positiva ($r \approx 0{,}694$) e uma influência estatisticamente significativa do IDD sobre o Conceito Enade. O coeficiente de determinação ajustado ($R^2 \approx 0{,}481$) indica que, embora o IDD seja um preditor relevante, ele explica menos da metade da variabilidade do Conceito Enade. Isso sugere que, apesar da significância estatística, o IDD não é o único fator determinante do desempenho, havendo influência de outras variáveis.

A constatação de que $51{,}9\%$ da variação do Conceito Enade não é explicada pelo IDD aponta para a principal limitação deste modelo de regressão. Evidencia-se a necessidade de investigações futuras que incorporem outras variáveis (como as dimensões do próprio CPC: infraestrutura, qualificação docente e percepção discente) para construir um modelo preditivo mais abrangente e explicativo.

Em síntese, este estudo reforça a natureza probabilística da inferência estatística. A rejeição das hipóteses nulas nas quatro análises realizadas não representa uma prova absoluta, mas sim fortes evidências contra $H_0$, considerando o nível de significância de 5\%. Os resultados obtidos devem ser interpretados como indícios de tendências consistentes, e não como relações determinísticas, destacando a importância de diferenciar significância estatística de causalidade e de relevância prática.
