\section{Resultados e Discussões}

Nesta seção iremos apresentar as análises estatísticas realizadas com base nas variáveis selecionadas na seção de Materiais e Métodos. 
Inicialmente, serão discutidas as estatísticas descritivas das variáveis quantitativas e qualitativas, seguido pelos testes de hipóteses levantados na seção de Objetivos.

\subsection{Análise das Estatísticas Descritivas}

Observando a base de dados selecionada e suas medições descritivas calculadas sobre as amostras feitas, é possível extrair algumas características das variáveis de interesse analisadas. As variáveis quantitativas, por exemplo, tem uma distribuição perceptivelmente semelhante à normal, com uma concentração de valores em torno da média e gradativa redução em direção às extremidades, porém cada uma possui suas particularidades. 

O Conceito Enade, com média 2,4753 e mediana 2,4753, apresenta certa inclinação a ter valores na parte inferior do intervalo (abaixo de 2,5), com seu desvio padrão de 0,9464 indicando a presença de valores afastados da média influenciando no seu cálculo. O IDD, por sua vez, tem um comportamento semelhante, porém invertido, com média 2.6829, mediana 2.661 e desvio padrão 1.0202 indicando a sua inclinação à parte superior do intervalo. 

O CPC possui uma leve inclinação para a parte de baixo do intervalo (menor que 3), com média 2.8335 e mediana 2.8249, e também concentra seus valores mais ao redor da média, como indicado por seu desvio padrão de 0.6381, indicando menos valores nas extremidades. 

Essas inclinações, porém, são muito leves para terem uma influência significativa nos intervalos de confiança de $\gamma=95\%$ para a média dessas variáveis, sendo que os de Conceito Enade (2.405, 2.5456) e IDD (2.6076, 2.7582) compartilham uma extensão semelhante aos de CPC (2.7727, 2.8942), mostrando um desempenho semelhante ao aproximar o valor estimado a partir de uma amostra de tamanho particular para cada variável, o que também influencia noa melhora das estimativas por levar em conta parâmetros como a variância.

Já ao observarmos as variáveis qualitativas, a modalidade de ensino presencial domina fortemente sobre a educação a distância, com cerca de 77\% dos cursos. Enquanto isso, as categorias administrativas públicas concentram por volta de 6,7\% das instituições de ensino. Nessas variáveis é possível perceber que os intervalos de confiança de 95\% para estimar suas proporções são mais extensos que os das quantitativas, sendo (72.5957, 82.071) para modalidade de ensino Presencial e (2.6748, 10.6585) para categorías administrativas  Públicas, o que pode ter influência da maximização da amostra pelo valor de p.

O \textit{script} R utilizado para calcular os intervalos pode ser observado no anexo B.

\subsection{Testes de Hipótese e Resultados}

\subsubsection{Hipótese sobre a Média}

O objetivo da análise foi comparar a média do Conceito Preliminar de Curso (CPC) entre cursos com Ensino Presencial e cursos com Ensino a Distância (EaD) em Santa Catarina. A variável utilizada foi o CPC contínuo, enquanto a modalidade de ensino identificou o tipo de curso. As hipóteses formuladas foram:

\begin{itemize}
    \item Hipótese nula (H$_0$): A média do CPC em cursos com ensino presencial em Santa Catarina é igual a média do CPC em cursos com ensino a distância (i.e., $\mu_\text{Presencial} = \mu_\text{EaD}$).

	\item Hipótese alternativa (H$_1$): A média do CPC em cursos com ensino presencial em Santa Catarina é maior que a média do CPC em cursos com ensino a distância (i.e., $\mu_\text{Presencial} > \mu_\text{EaD}$).
\end{itemize}

Para a análise, foi selecionada uma amostra aleatória de $n = 426$ cursos pertencentes às duas modalidades de interesse. Em seguida, foram extraídos os valores do CPC contínuo para cada grupo, e aplicou-se o teste t de Welch, que não pressupõe igualdade de variâncias entre as amostras. Para isso, utilizou-se a função t.test do R, com o parâmetro \texttt{var.equal = FALSE}, nível de confiança de 95\% (\texttt{conf.level = 0.95})  -- assumindo um nível de significância de 5\% -- e \texttt{alternative = "greater"}, aplicando assim um teste unilateral à direita, pois o interesse era verificar se as médias dos cursos presenciais eram significativamente maiores. O \textit{script} em R utilizado pode ser consultado no Anexo C.

Os resultados do teste indicaram $t = 4.6738$, graus de liberdade aproximados df $= 162.24$ e p-valor $= 3,09 \times 10^{-6}$. As médias amostrais foram 2,8942 para os cursos presenciais e 2,5898 para os cursos a distância. O intervalo de confiança unilateral de 95\% para a diferença das médias foi $[0,1967, +\infty)$, o que significa que, com 95\% de confiança, a média dos cursos presenciais excede a média dos cursos EaD em pelo menos aproximadamente 0,197 pontos. O gráfico da Figura \ref{fig:cpc_modalidade} ilustra essa diferença, mostrando uma distribuição de CPCs mais elevada entre os cursos presenciais.

Diante desses resultados, \textbf{rejeita-se a hipótese nula}, pois o p-valor é muito inferior ao nível de significância de 5\%, indicando que a diferença observada não é resultado do acaso. Conclui-se, portanto, que os cursos com ensino presencial apresentam CPC médio significativamente superior aos cursos com ensino a distância em Santa Catarina. Esse resultado sugere um desempenho médio mais elevado dos cursos presenciais segundo o indicador CPC.

\begin{figure}[h]
    \centering
    \caption{Distribuição do Conceito Preliminar de Curso (CPC) por Modalidade de Ensino.}
    \includegraphics[width=0.55\textwidth]{graficos/cpc_presencial_ead.png}
    \label{fig:cpc_modalidade}
\end{figure}

\subsubsection{Hipótese sobre a Proporção}

O objetivo desta análise foi verificar se a proporção de instituições federais em Santa Catarina é de 25\%, conforme afirma a hipótese nula. A variável analisada foi a \textit{categoria administrativa}, sendo considerado ``sucesso'' cada instituição classificada como Pública Federal. As hipóteses formuladas foram:

\begin{itemize}
    \item Hipótese nula (H$_0$): A proporção de instituições federais em Santa Catarina é igual a 25\% (i.e., $p = 0,25$).
    \item Hipótese alternativa (H$_1$): A proporção de instituições federais em Santa Catarina é diferente de 25\% (i.e., $p \neq 0,25$).
\end{itemize}

Foi selecionada uma amostra aleatória simples de $n = 150$ instituições, das quais 8 foram identificadas como federais, resultando em uma proporção amostral de $\hat{p} = 0,0533$. Aplicou-se o teste de proporção única (\texttt{prop.test} no R) sem correção de continuidade e com nível de confiança de 95\%. O resultado indicou uma estatística qui-quadrado de $\chi^2 = 30,942$, equivalente a uma estatística padronizada $z = -5,56$, com \textit{p}-valor de aproximadamente $2,66 \times 10^{-8}$. O intervalo de confiança de 95\% para a proporção verdadeira foi estimado em $[0,027; 0,102]$.

Como o valor calculado de $z = -5,56$ está muito além do limite crítico bilateral de $z_\text{crítico} = \pm 1,96$, o ponto amostral situa-se claramente dentro da região de rejeição de $H_0$. Essa relação pode ser visualizada na Figura~\ref{fig:testebilateral}, que mostra a distribuição normal padrão utilizada no teste, destacando as regiões críticas em vermelho e o valor observado de $z$ em azul.

Dado que o \textit{p}-valor é muito inferior ao nível de significância de 5\%, \textbf{rejeita-se a hipótese nula}, concluindo que há evidências estatísticas de que a proporção de instituições federais em Santa Catarina é significativamente diferente de 25\%. Observa-se, contudo, que a proporção observada (5,3\%) é consideravelmente menor do que o valor hipotetizado, indicando uma sub-representação de instituições federais no estado. Em termos absolutos, a diferença entre a proporção observada e a esperada é de aproximadamente 0,20, ou seja, 19,7 pontos percentuais a menos.

Do ponto de vista prático, esse resultado sugere que a participação das instituições federais em Santa Catarina é inferior ao quarto esperado, o que pode refletir características estruturais do sistema de ensino superior estadual, marcado por forte presença de instituições privadas. Apesar da robustez estatística do teste, é importante considerar que o número de instituições federais observadas (apenas oito) é pequeno, o que pode reduzir a precisão das estimativas. Ainda assim, o intervalo de confiança exclui claramente o valor de 25\%, reforçando a conclusão de que a proporção real é significativamente menor do que a hipotetizada.

\begin{figure}[H]
    \centering
    \caption{Distribuição normal padrão do teste bilateral para a proporção de instituições federais ($p = 0,25$), destacando as regiões críticas ($\alpha = 0,05$) e o valor observado de $z = -5,56$.}
    \includegraphics[width=0.85\textwidth]{graficos/teste_bilateral.pdf}
    \label{fig:testebilateral}
\end{figure}

\subsubsection{Hipótese sobre a Correlação}

\begin{itemize}
    \item Hipótese nula (H$_0$): Não há correlação entre o IDD e o Conceito Enade
    \item Hipótese Alternativa (H$_1$): Existe correlação entre IDD e Conceito Enade
\end{itemize}

Para testar a Hipótese, primeiro foram tomadas as tabelas de IDD e CPC filtradas para o estado de Santa Catarina e criadas amostras independentes de semente $= 123$ e $n_\text{CE} = 698$ e $n_\text{IDD} = 706$. 

As entradas foram associadas por meio do código da instituição, código do curso e ano em uma única tabela. Observou-se a presença de valores nulos, o que precisaram ser descartados já que não poderiam ser trabalhados no teste. Ao final, obteve-se uma tabela resultante dos valores de IDD e Conceito Enade relacionados para cada entrada, com um total de 388 entradas.

Realizando o teste de correlação de Pearson em R considerando 5\% de significância ($\alpha$), obteve-se os seguinte resultados: O Coeficiente de Correlação de Pearson (r) $= 0.6945254$ e o p-valor $<2.2\times 10^{-16}$, com grau de liberdade de 386, valor de t calculado igual a 18.966 e intervalo de confiança de 95\% igual a $[0.6391626, 0.7427272]$.

Observou-se uma forte correlação positiva do coeficiente de correlação entre IDD e Conceito Enade ao estar próximo de $r = 1$, além de estar dentro do intervalo de confiança calculado. O p-valor se mostrou muito inferior ao nível de significância adotado (0.05), o que leva a rejeição da hipótese nula H0, ou seja, existe sim uma correlação significativa entre os valores dos indicadores de IDD e o Conceito Enade.

\subsubsection{Hipótese sobre a Regressão}

\begin{itemize}
    \item Hipótese nula (H$_0$): O IDD não tem influência significativa sobre o Conceito Enade
	\item Hipótese Alternativa (H$_1$): O IDD tem influência significativa sobre o Conceito Enade
\end{itemize}

Para testar a hipótese, deve-se primeiro definir o modelo. O modelo de regressão linear ajustado é:

\[
\text{Conceito Enade} = \beta_0 + \beta_1 \times (\text{IDD}) + \varepsilon
\]

A hipótese nula será aceita caso o valor de $\beta_1$ seja igual a 0, ou seja, o valor do IDD não tem influência sobre o valor do Conceito Enade.

Utilizando o R, analisa-se a regressão e calcula-se os parâmetros utilizando os dados presentes na tabela resultante das amostras de IDD e Conceito Enade. Os resultados obtidos foram: intercepto ($\beta_0 = 1{,}00395$) e coeficiente angular ($\beta_1 = 0{,}59984$), erro padrão residual de $0{,}5829$ e \textit{p-valor} $< 2{,}2 \times 10^{-16}$, com o coeficiente de determinação ajustado ($R^2$) igual a $0{,}481$.
Na Figura 2, observa-se o gráfico da relação entre as amostras de IDD e Conceito Enade.

Percebe-se que o coeficiente angular estimado é de 0.59984, valor significativamente diferente de 0 como propõe a hipótese, além disso, o p-valor calculado se manteve abaixo do nível de significância de 5\%, portanto pode-se rejeitar a hipótese nula H0, concluindo que existe sim a influência de IDD sobre o valor do Conceito Enade.

Além disso, pode-se também tirar conclusões adicionais quanto ao coeficiente de determinação ajustado (0.481), indicando que  aproximadamente 48,1\% da variação no valor do Conceito Enade é explicada pelo IDD, sugerindo que cursos de valores maiores de IDD tendem a apresentar melhores valores do Conceito Enade.


\begin{figure}[H]
    \centering
    \caption{Relação de regressão linear entre IDD e Conceito Enade.}
    \includegraphics[width=0.65\textwidth]{graficos/Relacao_linear_entre_IDD_e_ENADE.png}
    \label{fig:regressao}
\end{figure}
