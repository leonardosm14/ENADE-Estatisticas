\section{Anexos - Códigos Fonte}

Todos os códigos, bases de dados, gráficos e tabelas podem ser observados em detalhes no repositório: \url{https://github.com/leonardosm14/ENADE-Estatisticas}

\begin{callblocklong}{Anexo A: Amostragem}
# --- Diretório ---
setwd("~/Documentos/UFSC/ENADE-Estatisticas")

# Carrega os dados filtrados de Santa Catarina
source(file = "src/script_sc.r")

# --- Parâmetros fixos padrão ---
z <- 1.96     # nível de confiança de 95%
d <- 0.05     # erro amostral tolerável (5%)

# --- Variáveis Quantitativas ---

# --- Funções ---
n0 <- function(variancia) {
  return((z^2 * variancia) / (d^2))
}

n <- function(n0, N) {
  return(n0 / (1 + (n0 / N)))
}

# --- 1. Conceito ENADE (Contínuo) ---
variancia_enade <- var(data_CPC_SC$conceito_enade_.continuo., na.rm = TRUE)
N_enade <- nrow(data_CPC_SC)

n0_enade <- n0(variancia_enade)
n_enade <- n(n0_enade, N_enade)

cat("\n--- Conceito ENADE (Contínuo) ---\n")
cat("Variância:", round(variancia_enade, 4), "\n")
cat("N:", N_enade, "\n")
cat("n0:", round(n0_enade, 2), "\n")
cat("n:", ceiling(n_enade), "\n")

# --- 2. IDD (Contínuo) ---
variancia_idd <- var(data_IDD_SC$idd_.continuo., na.rm = TRUE)
N_idd <- nrow(data_IDD_SC)

n0_idd <- n0(variancia_idd)
n_idd <- n(n0_idd, N_idd)

cat("\n--- IDD (Contínuo) ---\n")
cat("Variância:", round(variancia_idd, 4), "\n")
cat("N:", N_idd, "\n")
cat("n0:", round(n0_idd, 2), "\n")
cat("n:", ceiling(n_idd), "\n")

# --- 3. CPC (Contínuo) ---
variancia_cpc <- var(data_CPC_SC$cpc_.continuo., na.rm = TRUE)
N_cpc <- nrow(data_CPC_SC)

n0_cpc <- n0(variancia_cpc)
n_cpc <- n(n0_cpc, N_cpc)

cat("\n--- CPC (Contínuo) ---\n")
cat("Variância:", round(variancia_cpc, 4), "\n")
cat("N:", N_cpc, "\n")
cat("n0:", round(n0_cpc, 2), "\n")
cat("n:", ceiling(n_cpc), "\n")

# --- 4. IGC (Contínuo) ---
variancia_igc <- var(data_IGC_SC$igc_.continuo., na.rm = TRUE)
N_igc <- nrow(data_IGC_SC)

n0_igc <- n0(variancia_igc)
n_igc <- n(n0_igc, N_igc)

cat("\n--- IGC (Contínuo) ---\n")
cat("Variância:", round(variancia_igc, 4), "\n")
cat("N:", N_igc, "\n")
cat("n0:", round(n0_igc, 2), "\n")
cat("n:", ceiling(n_igc), "\n")

# --- Variáveis Qualitativas ---

p <- 0.5

# n0 é comum a todos

n0 <- z^2 * p * (1-p) / d^2

# --- 1. Categoria Administrativa ---
N_cat <- nrow(data_IGC_SC)
n_cat <- n(n0, N_cat)

cat("\n--- Categoria Administrativa ---\n")
cat("N:", N_cat, "\n")
cat("n:", ceiling(n_cat), "\n")


# --- 2. Modalidade de Ensino ---

N_mod <- nrow(data_CPC_SC)   # total de cursos (cada curso tem uma modalidade)
n_mod <- n(n0, N_mod)

cat("\n--- Modalidade de Ensino ---\n")
cat("N:", N_mod, "\n")
cat("n:", ceiling(n_mod), "\n")

\end{callblocklong}

\begin{callblocklong}{Anexo B: Análise Descritiva}
# --- Pacotes ---
library("dplyr")

# Carrega os dados filtrados de Santa Catarina
source(file = "script_sc.r")

# --- Filtrando os dados ---

# ENADE
dados_enade <- data_CPC_SC %>%
  select(conceito_enade_.continuo.) %>%
  filter(!is.na(conceito_enade_.continuo.))

# CPC
dados_cpc <- data_CPC_SC %>%
  select(cpc_.continuo.) %>%
  filter(!is.na(cpc_.continuo.))

# IDD
dados_idd <- data_IDD_SC %>%
  select(idd_.continuo.) %>%
  filter(!is.na(idd_.continuo.))

# IGC
dados_igc <- data_IGC_SC %>%
  select(igc_.continuo.) %>%
  filter(!is.na(igc_.continuo.))

# Modalidade de Ensino
dados_me <- data_CPC_SC %>%
  select(modalidade_de_ensino) %>%
  filter(!is.na(modalidade_de_ensino))

# Categoria Administrativa
dados_ca <- data_IGC_SC %>%
  select(categoria_administrativa) %>%
  filter(!is.na(categoria_administrativa))

# --- Criar amostra aleatória de tamanho n = 426, conforme "amostragem.r" ---

# ENADE
set.seed(123)  # para reprodutibilidade
amostra_enade <- dados_enade %>%
  sample_n(size = 699, replace = FALSE)

#CPC
set.seed(123)  # para reprodutibilidade
amostra_cpc <- dados_cpc %>%
  sample_n(size = 426, replace = FALSE)

# IDD
set.seed(123)  # para reprodutibilidade
amostra_idd <- dados_idd %>%
  sample_n(size = 707, replace = FALSE)

# IGC
set.seed(123)  # para reprodutibilidade
amostra_igc <- dados_igc %>%
  sample_n(size = 163, replace = FALSE)

# Modalidade de Enino
set.seed(123)  # para reprodutibilidade
amostra_me <- dados_me %>%
  sample_n(size = 300, replace = FALSE)

# Categoria Administrativa
set.seed(123)  # para reprodutibilidade
amostra_ca <- dados_ca %>%
  sample_n(size = 150, replace = FALSE)

# --- Estatísticas Descritivas ---

# ENADE
media_enade <- mean(amostra_enade$conceito_enade_.continuo.)
mediana_enade <- median(amostra_enade$conceito_enade_.continuo.)
dp_enade <- sd(amostra_enade$conceito_enade_.continuo.)
erro_enade = qt((1+0.95)/2, 699-1) * (dp_enade/(sqrt(699)))

cat("Média Conceito Enade:", round(media_enade, 4), "\n")
cat("Mediana Conceito Enade:", round(media_enade, 4), "\n")
cat("Desvio Padrão Conceito Enade:", round(dp_enade, 4), "\n")
cat("Intervalo de Confiança da Média do Conceito Enade:", round(media_enade-erro_enade, 4), "-", round(media_enade+erro_enade, 4), "\n")

# CPC
media_cpc <- mean(amostra_cpc$cpc_.continuo.)
mediana_cpc <- median(amostra_cpc$cpc_.continuo.)
dp_cpc <- sd(amostra_cpc$cpc_.continuo.)
erro_cpc = qt((1+0.95)/2, 426-1) * (dp_cpc/(sqrt(426)))

cat("Média CPC:", round(media_cpc, 4), "\n")
cat("Mediana CPC:", round(mediana_cpc, 4), "\n")
cat("Desvio Padrão CPC:", round(dp_cpc, 4), "\n")
cat("Intervalo de Confiança da Média do CPC:", round(media_cpc-erro_cpc, 4), "-", round(media_cpc+erro_cpc, 4), "\n")

# IDD
media_idd <- mean(amostra_idd$idd_.continuo.)
mediana_idd <- median(amostra_idd$idd_.continuo.)
dp_idd <- sd(amostra_idd$idd_.continuo.)
erro_idd = qt((1+0.95)/2, 707-1) * (dp_idd/(sqrt(707)))

cat("Média IDD:", round(media_idd, 4), "\n")
cat("Mediana IDD:", round(mediana_idd, 4), "\n")
cat("Desvio Padrão IDD:", round(dp_idd, 4), "\n")
cat("Intervalo de Confiança da Média do IDD:", round(media_idd-erro_idd, 4), "-", round(media_idd+erro_idd, 4), "\n")

# IGC
media_igc <- mean(amostra_igc$igc_.continuo.)
mediana_igc <- median(amostra_igc$igc_.continuo.)
dp_igc <- sd(amostra_igc$igc_.continuo.)
erro_igc = qt((1+0.95)/2, 163-1) * (dp_igc/(sqrt(163)))

cat("Média IGC:", round(media_igc, 4), "\n")
cat("Mediana IGC:", round(mediana_igc, 4), "\n")
cat("Desvio Padrão IGC:", round(dp_igc, 4), "\n")
cat("Intervalo de Confiança da Média do IGC:", round(media_igc-erro_igc, 4), "-", round(media_igc+erro_igc, 4), "\n")

# Modalidade de ensino
presencial <- sum(amostra_me$modalidade_de_ensino == "Educação Presencial")
total <- nrow(amostra_me)
prop_me <- (presencial / total)
erro_me <- qnorm((1+0.95)/2)*(sqrt(prop_me*(1-prop_me)/300))

cat("Proporção de Cursos Presenciais:", round(prop_me*100, 4), "\n")
cat("Intervalo de Confiança da Proporção de Cursos Presenciais:", round((prop_me-erro_me)*100, 4), "-", round((prop_me+erro_me)*100, 4), "\n")

# Categoria Adiministrativa
publicas <- sum(amostra_ca$categoria_administrativa == "Pública Federal",amostra_ca$categoria_administrativa == "Pública Estadual",amostra_ca$categoria_administrativa == "Pública Munucipal")
total <- nrow(amostra_ca)
prop_ca <- (publicas / total)
erro_ca <- qnorm((1+0.95)/2)*(sqrt(prop_ca*(1-prop_ca)/150))

cat("Proporção de Instituições Públicas:", round(prop_ca*100, 4), "\n")
cat("Intervalo de Confiança da Proporção de Instituições Públicas:", round((prop_ca-erro_ca)*100, 4), "-", round((prop_ca+erro_ca)*100, 4), "\n")

\end{callblocklong}


\begin{callblocklong}{Anexo C: Hipótese sobre a Média}
setwd("~/Documents/ENADE-Estatisticas")

# --- Pacotes ---
library(dplyr)
library(ggplot2)

# --- Filtrando os dados ---
dados_modalidade <- data_CPC_SC %>%
  filter(modalidade_de_ensino %in% c("Educação Presencial", "Educação a Distância")) %>%
  select(modalidade_de_ensino, cpc_.continuo.) %>%
  filter(!is.na(cpc_.continuo.))

# --- Criar amostra aleatória de tamanho n = 426, conforme "amostragem.r" ---
set.seed(123)  # para reprodutibilidade
dados_amostra <- dados_modalidade %>%
  sample_n(size = 426, replace = FALSE)

# --- Separando os grupos ---
presencial <- dados_amostra %>%
  filter(modalidade_de_ensino == "Educação Presencial") %>%
  pull(cpc_.continuo.)

ead <- dados_amostra %>%
  filter(modalidade_de_ensino == "Educação a Distância") %>%
  pull(cpc_.continuo.)

# --- Teste t de Welch ---
teste_t <- t.test(presencial, ead,
                  alternative = "greater",   # teste unilateral à direita
                  var.equal = FALSE)         # Welch (variâncias diferentes)

# --- Exibir resultado ---
print(teste_t)

# --- Boxplot comparativo ---
dev.new()
pdf("boxplot_modalidade.pdf", width = 7, height = 5)
print(
  ggplot(dados_amostra, aes(x = modalidade_de_ensino, y = cpc_.continuo., fill = modalidade_de_ensino)) +
    geom_boxplot(alpha = 0.7) +
    stat_summary(fun = mean, geom = "point", shape = 23, size = 3, fill = "white") +
    labs(x = "Modalidade de Ensino",
         y = "CPC (Contínuo)") +
    theme_minimal() +
    theme(legend.position = "none")
)
dev.off()   

\end{callblocklong}

\begin{callblocklong}{Anexo D: Hipótese sobre a Proporção}
setwd("~/Documents/ENADE-Estatisticas")
source(file = "src/script_sc.r")

# --- Pacotes ---
library(dplyr)

# --- Amostra aleatória de instituições ---
set.seed(123)
amostra_n <- 150
amostra_instituicoes <- data_IGC_SC %>%
  sample_n(amostra_n)

# --- Contagem de instituições federais ---
sucesso <- sum(amostra_instituicoes$categoria_administrativa == "Pública Federal")
total <- nrow(amostra_instituicoes)
prop_observada <- sucesso / total

cat("Tamanho da amostra:", total, "\n")
cat("Número de instituições federais na amostra:", sucesso, "\n")
cat("Proporção observada:", round(prop_observada, 3), "\n\n")

# --- Teste de hipótese para proporção única (bilateral) ---
# H0: p = 0.25
# H1: p != 0.25
teste_prop <- prop.test(x = sucesso, n = total, p = 0.25, alternative = "two.sided", correct = FALSE)
print(teste_prop)

# --- Interpretação ---
cat("\nInterpretação:\n")
if (teste_prop$p.value < 0.05) {
  cat("Rejeita-se H0: A proporção de instituições federais é significativamente diferente de 25%.\n")
} else {
  cat("Não se rejeita H0: Não há evidências de que a proporção de instituições federais seja diferente de 25%.\n")
}

# --- Gráfico do teste bilateral e salvamento em PDF ---
p0 <- 0.25
phat <- prop_observada
n <- total
z <- (phat - p0) / sqrt(p0 * (1 - p0) / n)
print(z)
alpha <- 0.05
zcrit <- qnorm(1 - alpha / 2)
print(zcrit)

# Caminho de saída
pdf("teste_bilateral.pdf", width = 10, height = 5)

# Dados da distribuição normal padrão
x <- seq(-6, 6, length = 1000)
y <- dnorm(x)

# Plot principal
plot(x, y, type = "l", lwd = 2, col = "black",
     main = "Teste Bilateral para Proporção",
     xlab = "Estatística Z", ylab = "Densidade de probabilidade")

# Regiões críticas
polygon(c(x[x >= zcrit], rev(x[x >= zcrit])),
        c(y[x >= zcrit], rep(0, sum(x >= zcrit))),
        col = rgb(1, 0, 0, 0.4), border = NA)
polygon(c(x[x <= -zcrit], rev(x[x <= -zcrit])),
        c(y[x <= -zcrit], rep(0, sum(x <= -zcrit))),
        col = rgb(1, 0, 0, 0.4), border = NA)

# Linha do valor observado
abline(v = z, col = "blue", lwd = 2)
abline(v = c(-zcrit, zcrit), col = "red", lty = 2)

# Legenda e texto
legend("topright",
       legend = c("Z observado", "Regioes criticas (alpha = 0.05)"),
       col = c("blue", "red"), lwd = 2, lty = c(1, 2), bty = "n")


text(z, dnorm(z) + 0.02, labels = sprintf("z = %.2f", z),
     col = "blue", pos = ifelse(z < 0, 2, 4))

dev.off()
\end{callblocklong}

\begin{callblocklong}{Anexo E: Hipótese sobre a Correlação e Regressão}
# Diretório da base de dados - Talvez precisa alterar, dependendo de onde o repositório estiver clonado.
setwd("~/Documents/ENADE-Estatisticas")

# ---- Filtragem das tabelas ----

# Retira eventuais entradas nulas
data_CPC_SC_limpo <- na.omit(data_CPC_SC[, c("ano", "codigo_da_ies", "nome_da_ies", "codigo_do_curso", "area_de_avaliacao", "conceito_enade_.continuo.")])
data_IDD_SC_limpo <- na.omit(data_IDD_SC[, c("ano", "codigo_da_ies", "nome_da_ies", "codigo_do_curso", "area_de_avaliacao", "idd_.continuo.")])

# Faz a amostragem de acordo com os valores de n calculados em "amostragem.r"
# Note que as amostras são independentes
set.seed(123)
amostra_CPC <- data_CPC_SC_limpo[sample(1:nrow(data_CPC_SC_limpo), 699, replace = FALSE), ]
rownames(amostra_CPC) <- NULL
amostra_IDD <- data_IDD_SC_limpo[sample(1:nrow(data_IDD_SC_limpo), 707, replace = FALSE), ]
rownames(amostra_IDD) <- NULL

# Faz o merge das tabelas pelas colunas-chave
data_merged <- merge(
  x = amostra_CPC,
  y = amostra_IDD,
  by = c("ano", "codigo_da_ies", "codigo_do_curso"),
  suffixes = c("_CPC", "_IDD")
)

# Seleciona as colunas principais para análise
data_corre_reg <- data.frame(
  ano = data_merged$ano,
  codigo_da_ies = data_merged$codigo_da_ies,
  codigo_do_curso = data_merged$codigo_do_curso,
  nome_da_ies = data_merged$nome_da_ies_CPC,
  nome_do_curso = data_merged$area_de_avaliacao_CPC,
  conceito_enade = data_merged$conceito_enade_.continuo.,
  idd = data_merged$idd_.continuo.
)

# Remove linhas com valores faltantes (NA) em Enade ou IDD
data_corre_reg <- subset(data_corre_reg, !is.na(conceito_enade) & !is.na(idd))
rownames(data_corre_reg) <- NULL


# ---- Teste de hipótese sobre correlação ----


# Teste de correlação entre ENADE e IDD
cor.test(data_corre_reg$conceito_enade, data_corre_reg$idd, method = "pearson")


# ---- Teste de hipótese sobre regressão ----


# Modelo de regressão linear (ENADE dependente de IDD)
modelo <- lm(conceito_enade ~ idd, data = data_corre_reg)

# Sumário com teste t dos coeficientes
summary(modelo)

# Gráfico da Relação ENADE e IDD contínuos
dev.new()
plot(data_corre_reg$idd, data_corre_reg$conceito_enade,
     xlab = "IDD", ylab = "Conceito ENADE",
     main = "Relação linear entre IDD e ENADE",
     pch = 19, col = "blue")

#Plot da Linha de regressão
abline(modelo, col = "red", lwd = 2)

\end{callblocklong}