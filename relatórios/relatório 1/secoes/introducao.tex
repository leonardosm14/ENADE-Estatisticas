\section{Introdução}

O \textbf{Exame Nacional de Desempenho dos Estudantes (ENADE)} foi instituído em 2004 pelo Instituto Nacional de Estudos e Pesquisas Educacionais Anísio Teixeira (INEP) com o objetivo de avaliar o desempenho dos concluintes de cursos de graduação no Brasil. A cada ciclo trienal, determinados cursos em todo o país realizam a prova, que avalia o conhecimento dos estudantes sobre os conteúdos programáticos definidos nas diretrizes curriculares de cada curso \cite{seavi_enade}.  

O ciclo trienal do ENADE é a forma de organização do exame em períodos de três anos, de modo que diferentes cursos de graduação sejam avaliados em anos distintos, garantindo a cobertura de todas as áreas do conhecimento de forma rotativa. Cada ano do ciclo avalia grupos específicos de cursos, conforme suas áreas de atuação, conforme descrito na Tabela \ref{tab:ciclo-trienal}.

\begin{table}[H]
\centering
\caption{Distribuição dos cursos avaliados por ano do ciclo trienal do ENADE}
\label{tab:ciclo-trienal}
\vspace{0.2cm}
\resizebox{0.65\textwidth}{!}{%
\begin{tabular}{>{\centering\arraybackslash}m{2cm}|p{12cm}}
\hline
\textbf{Ano} & \textbf{Cursos Avaliados} \\
\hline
Ano I & Cursos de bacharelado nas áreas de Ciências Agrárias, Ciências da Saúde e áreas afins.\\
& Cursos de bacharelado nas áreas de Engenharias e Arquitetura e Urbanismo.\\
& Cursos Superiores de Tecnologia nas áreas de Ambiente e Saúde, Produção Alimentícia, Recursos Naturais, Militar e Segurança. \\
\hline
Ano II & Cursos de bacharelado nas áreas de Ciências Biológicas, Ciências Exatas e da Terra, Linguística, Letras e Artes e áreas afins.\\
& Cursos de licenciatura nas áreas de Ciências da Saúde, Ciências Humanas, Ciências Biológicas, Ciências Exatas e da Terra, Linguística, Letras e Artes.\\
& Cursos de bacharelado nas áreas de Ciências Humanas e Ciências da Saúde, com cursos avaliados no âmbito das licenciaturas.\\
& Cursos Superiores de Tecnologia nas áreas de Controle e Processos Industriais, Informação e Comunicação, Infraestrutura e Produção Industrial. \\
\hline
Ano III & Cursos de bacharelado nas áreas de Ciências Sociais Aplicadas e áreas afins.\\
& Cursos de bacharelado nas áreas de Ciências Humanas e áreas afins que não tenham cursos avaliados no âmbito das licenciaturas.\\
& Cursos Superiores de Tecnologia nas áreas de Gestão e Negócios, Apoio Escolar, Hospitalidade e Lazer, Produção Cultural e Design. \\
\hline
\end{tabular}%
}
\\
\vspace{0.2cm}
\footnotesize Fonte: Adaptado de \cite{inep_enade}
\end{table}

O formato da avaliação pode variar de ano para ano, mas, considerando os dados mais recentes disponíveis que serão analisados, relativos à prova de 2023, conforme o edital, a avaliação foi composta por 40 questões: 10 de Formação Geral (FG) e 30 de Conhecimento Específico (CE) \cite{inep_edital_2023}. Além disso, os estudantes também respondem ao Questionário do Estudante, que coleta informações qualitativas sobre a organização didático-pedagógica, infraestrutura, instalações físicas da instituição e oportunidades de ampliação da formação acadêmica e profissional \cite{seavi_enade}.

A fim de avaliar o desempenho dos estudantes e a qualidade das instituições de ensino, o ENADE utiliza as provas e os questionários para calcular índices quantitativos e qualitativos, tais como o Conceito Enade (CE), o Conceito Preliminar de Curso (CPC), o Indicador de Diferença entre os Desempenhos Observados e Esperado (IDD) e o Índice Geral de Cursos (IGC), que serão detalhados na Seção de Materiais e Métodos, juntamente com a Categoria Administrativa e a Modalidade de Ensino dos cursos avaliados.

\subsection{Objetivos}

O presente relatório tem como objetivo apresentar uma análise estatística dos dados referentes aos índices de qualidade coletados pelo INEP no período de 2021 a 2023, permitindo observar características essenciais e atualizadas das instituições de ensino públicas e privadas, que se utilizam da modalidade presencial ou a distância, do estado de Santa Catarina. Para isso, serão calculadas estatísticas de tendência central e de dispersão, além de análises gráficas, buscando evidenciar padrões e possibilitar conclusões relevantes a respeito das variáveis analisadas. 

\subsection{Objetivos Específicos}
\begin{enumerate}
    \item Definir e classificar de forma clara e objetiva as variáveis de interesse e as motivações para sua escolha; 
    \item Apresentar e interpretar gráficos e tabelas que tratam da frequência das variáveis de interesse;
    \item Dissertar sobre medidas de resumo aplicadas às variáveis de interesse;
    \item Expor a relação entre as variáveis de interesse, duas a duas.
\end{enumerate}